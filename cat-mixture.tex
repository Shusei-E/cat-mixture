\documentclass[12pt,letterpaper]{article}
\usepackage[margin = 1.1in]{geometry}

% Essential packages
\usepackage{color, colortbl}
\usepackage{graphicx}
\usepackage{enumitem}

% Parameters
\usepackage{hanging}
\usepackage{setspace}
\usepackage{lscape}
\usepackage{verbatim}
\usepackage{placeins}
\usepackage{floatpag}
\usepackage{subcaption}
\usepackage{pdfpages} % include pdf
\usepackage{abstract}
\usepackage{multicol}
\usepackage{caption}
\usepackage{lineno}
\usepackage{etoc}


% Figure caption?
\usepackage[labelfont=bf]{caption}
\setlength{\captionmargin}{10pt}

\usepackage{changepage} % adjustwidth

\usepackage{soul} % underline
\usepackage{tcolorbox}

% Minion for main text and math?
% \usepackage{MinionPro}

% Helvetica for sans serif
% (scaled to match size of Minion)
\usepackage[scaled=0.90]{helvet}

% Bera Mono for monospaced
% (scaled to match size of Minion)
\usepackage[T1]{fontenc}
\usepackage[scaled=0.8]{beramono}

% footnote
\usepackage{titling}
 %% Footnote font size setup
\usepackage[hang, bottom]{footmisc}
\setlength{\footnotemargin}{1.2em} % separation between number and text
\interfootnotelinepenalty=10000 %% Completely prevent breaking of footnotes


% spacing
\newcommand\spacingset[1]{\renewcommand{\baselinestretch}%
{#1}\small\normalsize}


% math
\usepackage{amsmath,amssymb, amsthm, mathtools}
\numberwithin{equation}{section}
\usepackage{bm}
\usepackage{dsfont}
\usepackage{bbding} % checkmark

\newcommand{\var}{\textnormal{Var}}
\newcommand{\cov}{\textnormal{Cov}}
\newcommand{\cor}{\textnormal{Cor}}
\newcommand{\bth}{\bm{\theta}}
\newcommand{\bmu}{\bm{\mu}}
\newcommand{\bpi}{\bm{\pi}}
\newcommand{\N}{\mathcal{N}}
\newcommand{\y}{\textsc{y}}
\newcommand{\x}{\textsc{x}}
\newcommand{\bx}{\mathbf{x}}
\newcommand{\by}{\mathbf{y}}
\newcommand{\bz}{\mathbf{z}}
\newcommand{\bX}{\mathbf{X}}
\newcommand{\bY}{\mathbf{Y}}
\newcommand{\bZ}{\mathbf{Z}}
\newcommand{\sR}{\textsc{r}}
\newcommand{\sL}{\textsc{l}}
\newcommand\norm[1]{\left\lVert#1\right\rVert} % norm enviornment


% Colored links
\definecolor{mygray}{gray}{0.9}
\definecolor{crimson}{RGB}{156,0,0}
\definecolor{red}{HTML}{E31A1C}
\definecolor{blue}{HTML}{1F78B4}
\usepackage{hyperref}
\hypersetup{colorlinks = true,
            citecolor = crimson,
            urlcolor = crimson,
            linkcolor = crimson,
            plainpages = false,
            bookmarksnumbered,
            pdfauthor = {}}
\usepackage{url}



% Titling
% \setlength{\droptitle}{-5.2em}
\pretitle{\begin{center}\large\bfseries}
\posttitle{\end{center}}
% \preauthor{\begin{flushleft}}
% \postauthor{\end{flushleft}}
% \predate{\begin{flushleft}\normalsize}
% \postdate{\end{flushleft}}

% Sectioning
\usepackage{titlesec}
\titleformat*{\section}{\normalsize\bfseries}
\titleformat*{\subsection}{\normalsize\itshape}


% tables
\usepackage{tabularx}
\usepackage{colortbl}
\definecolor{lgray}{gray}{0.85}
\newcolumntype{C}{>{\centering\arraybackslash}X}
\newcolumntype{G}{>{\columncolor{lgray}}C}
% Tabular X special row types
\newcolumntype{R}{>{\raggedleft\arraybackslash}X}
\newcolumntype{L}{>{\raggedright\arraybackslash}X}
\newcolumntype{m}{>{\hsize=1.40\hsize}X}
\newcolumntype{b}{>{\hsize=2.10\hsize}X}
\newcolumntype{s}{>{\hsize=0.20\hsize}C}

\usepackage{booktabs}
\usepackage{rotating}
\usepackage{arydshln}


% some break bug
\makeatletter
\def\blx@maxline{77}
\makeatother


% Code
\usepackage{listings}
\definecolor{dkgreen}{rgb}{0,0.6,0}
\definecolor{gray}{rgb}{0.5,0.5,0.5}
\definecolor{mauve}{rgb}{0.58,0,0.82}

\lstset{frame=tb,
  language=R,
  aboveskip=3mm,
  belowskip=3mm,
  showstringspaces=false,
  columns=flexible,
  basicstyle={\small\ttfamily},
  numbers=none,
  numberstyle=\tiny\color{gray},
  keywordstyle=\color{black},
  commentstyle=\color{dkgreen},
  stringstyle=\color{mauve},
  breaklines=true,
  breakatwhitespace=true,
  tabsize=3
}


% markdown -------

% Figures
\usepackage{graphicx,grffile}
\makeatletter
\def\maxwidth{\ifdim\Gin@nat@width>\linewidth\linewidth\else\Gin@nat@width\fi}
\def\maxheight{\ifdim\Gin@nat@height>\textheight\textheight\else\Gin@nat@height\fi}
\makeatother

% Scale images if necessary, so that they will not overflow the page
% margins by default, and it is still possible to overwrite the defaults
% using explicit options in \includegraphics[width, height, ...]{}
\setkeys{Gin}{width=\maxwidth,height=\maxheight,keepaspectratio}


% Shading and Rmd macros
\usepackage{color}
\usepackage{fancyvrb}
\newcommand{\VerbBar}{|}
\newcommand{\VERB}{\Verb[commandchars=\\\{\}]}
\DefineVerbatimEnvironment{Highlighting}{Verbatim}{commandchars=\\\{\}}
% Add ',fontsize=\small' for more characters per line
\usepackage{framed}
\definecolor{shadecolor}{RGB}{248,248,248}
\newenvironment{Shaded}{\begin{snugshade}}{\end{snugshade}}
\newcommand{\AlertTok}[1]{\textcolor[rgb]{0.94,0.16,0.16}{#1}}
\newcommand{\AnnotationTok}[1]{\textcolor[rgb]{0.56,0.35,0.01}{\textbf{\textit{#1}}}}
\newcommand{\AttributeTok}[1]{\textcolor[rgb]{0.77,0.63,0.00}{#1}}
\newcommand{\BaseNTok}[1]{\textcolor[rgb]{0.00,0.00,0.81}{#1}}
\newcommand{\BuiltInTok}[1]{#1}
\newcommand{\CharTok}[1]{\textcolor[rgb]{0.31,0.60,0.02}{#1}}
\newcommand{\CommentTok}[1]{\textcolor[rgb]{0.56,0.35,0.01}{\textit{#1}}}
\newcommand{\CommentVarTok}[1]{\textcolor[rgb]{0.56,0.35,0.01}{\textbf{\textit{#1}}}}
\newcommand{\ConstantTok}[1]{\textcolor[rgb]{0.00,0.00,0.00}{#1}}
\newcommand{\ControlFlowTok}[1]{\textcolor[rgb]{0.13,0.29,0.53}{\textbf{#1}}}
\newcommand{\DataTypeTok}[1]{\textcolor[rgb]{0.13,0.29,0.53}{#1}}
\newcommand{\DecValTok}[1]{\textcolor[rgb]{0.00,0.00,0.81}{#1}}
\newcommand{\DocumentationTok}[1]{\textcolor[rgb]{0.56,0.35,0.01}{\textbf{\textit{#1}}}}
\newcommand{\ErrorTok}[1]{\textcolor[rgb]{0.64,0.00,0.00}{\textbf{#1}}}
\newcommand{\ExtensionTok}[1]{#1}
\newcommand{\FloatTok}[1]{\textcolor[rgb]{0.00,0.00,0.81}{#1}}
\newcommand{\FunctionTok}[1]{\textcolor[rgb]{0.00,0.00,0.00}{#1}}
\newcommand{\ImportTok}[1]{#1}
\newcommand{\InformationTok}[1]{\textcolor[rgb]{0.56,0.35,0.01}{\textbf{\textit{#1}}}}
\newcommand{\KeywordTok}[1]{\textcolor[rgb]{0.13,0.29,0.53}{\textbf{#1}}}
\newcommand{\NormalTok}[1]{#1}
\newcommand{\OperatorTok}[1]{\textcolor[rgb]{0.81,0.36,0.00}{\textbf{#1}}}
\newcommand{\OtherTok}[1]{\textcolor[rgb]{0.56,0.35,0.01}{#1}}
\newcommand{\PreprocessorTok}[1]{\textcolor[rgb]{0.56,0.35,0.01}{\textit{#1}}}
\newcommand{\RegionMarkerTok}[1]{#1}
\newcommand{\SpecialCharTok}[1]{\textcolor[rgb]{0.00,0.00,0.00}{#1}}
\newcommand{\SpecialStringTok}[1]{\textcolor[rgb]{0.31,0.60,0.02}{#1}}
\newcommand{\StringTok}[1]{\textcolor[rgb]{0.31,0.60,0.02}{#1}}
\newcommand{\VariableTok}[1]{\textcolor[rgb]{0.00,0.00,0.00}{#1}}
\newcommand{\VerbatimStringTok}[1]{\textcolor[rgb]{0.31,0.60,0.02}{#1}}
\newcommand{\WarningTok}[1]{\textcolor[rgb]{0.56,0.35,0.01}{\textbf{\textit{#1}}}}


% Parmaters --
\title{ \Large\textbf{Probabilistic K-means with Multinomial Outcomes}}

\author{\normalsize  Shiro
Kuriwaki\thanks{Thanks to Shusei Eshima, Sooahn Shin, and Soichiro Yamauchi for their help.} }

\date{\normalsize October 2019}

\begin{document}

\maketitle



\section{Data Generating Process}

(Note: This first setup basically follows the Murphy text (2017),
section 11.2.2. Then it tries to implement it in stan.)

\paragraph{Setup}

Index individuals by \(i \in \{1:N\}\) and the universe of races
excluding the top of the ticket as \(j \in \{1:D\}\). The data we
observe is a length-\(D\) vector of votes \(\by_i\). \(y_{ij}\) is a
discrete response value which can be 1 of \(M\) possibilities.

\paragraph{Likelihood}

For now, let's use \(M = 2\) so that each vote \(y_{ij}\) be a
\emph{binary} variable for splitting their ticket or not. \(y_{ij} = 1\)
would mean voter \(i\) splitting their ticket in some office \(j\), with
reference to a top of the ticket office like the President or Governor.

Let \(\mu_{z[i], j} \in [0, 1]\) be the parameter that governs each
\(y_{ij}\). i.e., \(\Pr(y_{ij} = 1) = \mu_{z[i], j}.\) However, we won't
estimate a \(\mu\) for each of the \(N\) individuals; we will be
estimating only \(K\) sets of length-\(D\) vectors \(\bmu_{k}\).

Individual voters come from one of \(K\) different clusters.
Individual's cluster membership is denoted \(z[i]\). Therefore, we can
express the joint density as follows. By referring to the set of
parameters generically as \(\bth\),

\begin{align}
p(\by_{i} \mid z_{i} = k, \bth) = \prod^{D}_{j = 1} \text{Bern}(y_{ij} | \bmu_k) = \prod^{D}_{j = 1} \mu_{jk}^{y_{ij}}(1 - \mu_{jk})^{1 - y_{ij}}
\end{align}

On the log scale,

\begin{align}
\log p(\by_{i} \mid z_{i} = k, \bth) = \sum_{j=1}^{D}(y_{ij}\log\mu_{jk} + (1 - y_{ij})\log\mu_{jk})
\end{align}

\paragraph{Cluster membership}

The random variable \(z_i\) comes from a discrete distribution. We put a
discrete prior for this,

\begin{align}
p(z_i) &= \text{Cat}(\bth)
\end{align}

Where the length-\(K\) simplex \(\bth\) is called the mixing proportion.

The benefit of this modeling exercise over that from a naive sample of
\(N \times D\) Bernoullis is that we have captured the correlations
between
variables.\footnote{That dependency can be expressed as \(\mathds{E}(\by) = \sum_{k = 1}^{K} \theta_{k} \bmu_{k}\) and 
\(\cov({\by}) = \sum_{k} \theta_{k} (\mathbf{\Sigma}_k \bmu_k\bmu_{k}^{\top}) - \mathds{E}(\by)\mathds{E}(\by)^\top\), where \(\Sigma_k = \text{diag}(\mu_{jk}(1 - \mu_{jk}))\).}

\paragraph{Cluster probabilities}

From this likelihood we can extract the posterior probability a point
belongs to a cluster, \(p(z_i = k \mid \by_i, \bth)\). Some call this
the \emph{responsibility} of cluster \(k\) for point \(i\). It can be
computed by Bayes rule as:

\begin{align}
p(z_i = k \mid \by_{i}, \bth) &=  \frac{p(z_i = k \mid \bth)\cdot p(\by_{i} \mid z_i = k, \bth)}{\sum^{K}_{k^\prime = 1} p(\by_{i} \mid z_i = k^\prime, \bth) \cdot p(z_i = k^\prime \mid \bth)}\\
&\propto p(z_i = k \mid \bth)\cdot p(\by_{i} \mid z_i = k, \bth)\\
&=  \theta_k\cdot\prod^{D}_{j=1}\mu_{jk}^{y_{ij}}(1 - \mu_{jk})^{1 - y_{ij}}
\end{align}

On the log scale, the normalized probability is computed as follows.

\begin{align*}
&\log p(z_i = k \mid \by_{i}, \bth)\\
=& \log p(z_i = k \mid \bth) + \log p(\by_i \mid z_i = k,  \bth)   - \mathrm{log\_sum\_exp}_{k^\prime = 1}^K\left(\log p(z_i = k^\prime \mid \bth) + \log p(\by_i \mid z_i = k^\prime, \bth)\right)\\
\end{align*}

\pagebreak

\section{EM algorithm}

We can estimate the key parameters \(\bmu\) and \(\bth\) by
EM.\footnote{Thanks to Soichiro Yamauchi for deriving this algorithm in the original iteration.}

The complete log-likelihood can be writen as

\begin{align}
\mathds{L}_{\mathrm{comp}}(\bmu, \bth | \bY, \bZ) & \prod^{N}_{i=} \\
\end{align}

\section{Stan Code}

The code below tries to adapt the Stan user manual on
\href{https://mc-stan.org/docs/2_20/stan-users-guide/soft-k-means.html}{Soft-K
means} and
\href{https://mc-stan.org/docs/2_20/stan-users-guide/summing-out-the-responsibility-parameter.html}{Summing
out the responsibility mixture}. My current attempt at the Stan code:

\begin{Shaded}
\begin{Highlighting}[]
\NormalTok{data \{}
\NormalTok{  int}\OperatorTok{<}\NormalTok{lower=}\DecValTok{1}\OperatorTok{>}\StringTok{ }\NormalTok{K;                }\OperatorTok{/}\ErrorTok{/}\StringTok{ }\NormalTok{number of clusters}
\NormalTok{  int}\OperatorTok{<}\NormalTok{lower=}\DecValTok{1}\OperatorTok{>}\StringTok{ }\NormalTok{D;                }\OperatorTok{/}\ErrorTok{/}\StringTok{ }\NormalTok{number of offices}
\NormalTok{  int}\OperatorTok{<}\NormalTok{lower=}\DecValTok{1}\OperatorTok{>}\StringTok{ }\NormalTok{N;                }\OperatorTok{/}\ErrorTok{/}\StringTok{ }\NormalTok{number of voters}
\NormalTok{  int}\OperatorTok{<}\NormalTok{lower=}\DecValTok{0}\NormalTok{, upper=}\DecValTok{1}\OperatorTok{>}\StringTok{ }\NormalTok{y[N, D]; }\OperatorTok{/}\ErrorTok{/}\StringTok{ }\NormalTok{data}
\NormalTok{  vector}\OperatorTok{<}\NormalTok{lower=}\DecValTok{0}\OperatorTok{>}\NormalTok{[K] alpha;      }\OperatorTok{/}\ErrorTok{/}\StringTok{ }\NormalTok{hyperparameter }\ControlFlowTok{for}\NormalTok{ cluster prevalence}
\NormalTok{\}}

\NormalTok{transformed data \{}
\NormalTok{  real}\OperatorTok{<}\NormalTok{upper=}\DecValTok{0}\OperatorTok{>}\StringTok{ }\NormalTok{neg_log_K;}
\NormalTok{  neg_log_K =}\StringTok{ }\OperatorTok{-}\KeywordTok{log}\NormalTok{(K);}
\NormalTok{\}}


\NormalTok{parameters \{}
\NormalTok{  simplex[K] pi_mixture;          }\OperatorTok{/}\ErrorTok{/}\StringTok{ }\NormalTok{mixing proportions}
\NormalTok{  real}\OperatorTok{<}\NormalTok{lower=}\DecValTok{0}\NormalTok{,upper=}\DecValTok{1}\OperatorTok{>}\StringTok{ }\NormalTok{mu[K, D]; }\OperatorTok{/}\ErrorTok{/}\StringTok{ }\NormalTok{Bernoulli probability}
\NormalTok{\}}

\NormalTok{model \{}
  \OperatorTok{/}\ErrorTok{/}\StringTok{ }\NormalTok{prior}
\NormalTok{  pi_mixture }\OperatorTok{~}\StringTok{ }\KeywordTok{dirichlet}\NormalTok{(alpha);}
  \ControlFlowTok{for}\NormalTok{ (k }\ControlFlowTok{in} \DecValTok{1}\OperatorTok{:}\NormalTok{K) \{}
    \ControlFlowTok{for}\NormalTok{ (j }\ControlFlowTok{in} \DecValTok{1}\OperatorTok{:}\NormalTok{D) \{}
      \OperatorTok{/}\ErrorTok{/}\StringTok{ }\NormalTok{something with mode at a low pr}
\NormalTok{      mu[k, j] }\OperatorTok{~}\StringTok{ }\KeywordTok{beta}\NormalTok{(}\DecValTok{2}\NormalTok{, }\DecValTok{5}\NormalTok{);}
\NormalTok{    \}}
\NormalTok{  \}}

  \OperatorTok{/}\ErrorTok{/}\StringTok{ }\NormalTok{likelihood}
  \ControlFlowTok{for}\NormalTok{ (n }\ControlFlowTok{in} \DecValTok{1}\OperatorTok{:}\NormalTok{N) \{}
    \ControlFlowTok{for}\NormalTok{(j }\ControlFlowTok{in} \DecValTok{1}\OperatorTok{:}\NormalTok{D) \{}
      \OperatorTok{/}\ErrorTok{/}\StringTok{ }\NormalTok{initiate with z }\OperatorTok{~}\StringTok{ }\KeywordTok{Cat}\NormalTok{(}\DecValTok{1}\OperatorTok{/}\NormalTok{K, ... }\DecValTok{1}\OperatorTok{/}\NormalTok{K)}
\NormalTok{      vector[K] lps =}\StringTok{ }\KeywordTok{rep_vector}\NormalTok{(neg_log_K, K);}
      \ControlFlowTok{for}\NormalTok{ (k }\ControlFlowTok{in} \DecValTok{1}\OperatorTok{:}\NormalTok{K) \{}
        \OperatorTok{/}\ErrorTok{/}\StringTok{ }\NormalTok{sum all possible values of}
\NormalTok{        lps[k] }\OperatorTok{+}\ErrorTok{=}\StringTok{ }\KeywordTok{bernoulli_lpmf}\NormalTok{(y[n, j] }\OperatorTok{|}\StringTok{ }\NormalTok{mu[k, j]) }\OperatorTok{+}\StringTok{ }\KeywordTok{log}\NormalTok{(pi_mixture[k]);}
\NormalTok{      \}}
\NormalTok{      target }\OperatorTok{+}\ErrorTok{=}\StringTok{ }\KeywordTok{log_sum_exp}\NormalTok{(lps);}
\NormalTok{    \}}
\NormalTok{  \}}



\NormalTok{\}}
\end{Highlighting}
\end{Shaded}

\pagebreak

\appendix

The \textbf{Appendix} includes Old Iterations and specifications I will
incorporate later.

\section{Adding covariates}

Later on, we will model \(\theta_{jk}\) as a function of \(v_{j}\),
covariates of candidate \(j\).

\[\theta_{jk} = \frac{\exp(v_{j}^{\top}\beta_{jk})}{\sum_{j^{\prime}} \exp(v_{j^{\prime}}^{\top}\beta_{j^{\prime}k})} \]

For now, we will model three attributes of the candidate: whether the
candidate is an incumbent, and whether the candidate is in an open-seat.

For example, assume we are talking about Republican voter:

\begin{table}[!h]
\centering
\footnotesize
\begin{tabularx}{0.6\linewidth}{cLlCc}\toprule
 & & \multicolumn{2}{c}{Republican Candidate} & \\\cline{3-4}
\(j\) & Race & Name & Incumbent & Open-seat\\\midrule
1 & HD  15  & Samuel Rivers & 1 & 0 \\
2 & HD  94  & Con Chellis & 0 & 1 \\
3 & HD  99  & Nancy Mace & 1 & 0 \\
4 & HD 110  & William Cogswell & 1 & 0 \\
5 & HD 112  & Mike Sottile & 1 & 0 \\
6 & HD 114  & Lin Bennett & 1 & 0 \\
7 & HD 115  & Peter McCoy & 1 & 0 \\
8 & HD 117  & Bill Crosby  & 1 & 0 \\
9 & HD 119  & Paul Sizemore & 0 & 0 \\
\bottomrule
\end{tabularx}
\end{table}

\section{Model 2: Kosuke}

We model the outcome as coming from a categorical distribution:

\begin{align*}
(y_{ij} \mid Z_{i} = k) &\sim \text{Categorical}(\theta_{k, j})
\end{align*}

For example, suppose that there are three types of voters. They each
have a given value of \(\theta_{k, j}\): \begin{align*}
\begin{cases}
\text{Always straight} &\theta_{k, j} = (0.97, 0.01, 0.02) \text{ if } k = 1\\
\text{Always split} &\theta_{k, j} = (0.01, 0.97, 0.02) \text{ if } k = 2\\
\text{Random} &\theta_{k, j} = (0.49, 0.49, 0.02) \text{ if } k = 3\\
\end{cases}
\end{align*}

For simplicity, let's assume this holds regardless of the office, i.e.
\(\theta_{k, j} = \theta_{k, j^\prime} ~ \forall j \neq j^\prime\).

The cluster is also a categorical variable, \begin{align*}
Z_{i} &\sim \text{Categorical}(\psi),\\
\psi &\sim \text{Dirichlet}(\alpha)
\end{align*}

The length-\(K\) simplex \(\psi\) is called the \textbf{mixing
proportion}. This is a low-dimensional quantity of interest.

From substantive background knowledge, our prior is that most voters are
straight ticket voters, so we can set the hyperparameter to

\begin{align*}
\alpha = (2.0, 1.5, 1.0)
\end{align*}

\section{Model 2: Soichiro}

A simpler model may be to model the office type as a categorical
variable separately. Let this variable be \(W\). The main attribute of
this variable is its level of office. Our main interest is whether some
offices exhibit more split ticketting than others, so the simplest setup
is to let

\[w_{j} \in \{0, 1\}.\]

where the values indicate

\begin{table}[!h]
\centering
\begin{tabular}{ll}\toprule
0 & low propensity to generate split ticket\\
1 & high propensity to generate split ticket\\\bottomrule
\end{tabular}
\end{table}

Later on, we can add a variable for candidate level incumbency.

Then, the simplex governing the parameter can be indexed as

\begin{align*}
(Y_{ij} | Z_i = z, W_j = w) \sim \text{Categorical}(\theta_{z, w})
\end{align*}

The values of \(W_j\) for offices \(j \in \{1:J\}\) can be modeled as a
categorical variable, or more simply a Bernoulli

\begin{align*}
W \sim  \text{Bern}(\theta)
\end{align*}

where \(\theta \in [0, 1].\)
\end{document}

